\documentclass[14pt, a4paper, utf8, draft]{extarticle}
\usepackage[utf8]{inputenc}
\usepackage[english, russian]{babel}
\usepackage[T2A]{fontenc}
\usepackage{xltxtra}
\usepackage{xecyr}

\usepackage{fontspec}
\defaultfontfeatures{Scale=MatchLowercase}
\setmainfont[Mapping=tex-text]{Liberation Serif}
\setsansfont[Mapping=tex-text]{Liberation Sans}
\setmonofont{Droid Sans Mono}

\usepackage{geometry}
\geometry{left=2.5cm}
\geometry{right=2.5cm}
\geometry{top=2.5cm}
\geometry{bottom=2.5cm}

\begin{document}

{\Large Список публикаций}
\begin{enumerate}
\item Морозов А. С. Трассировка лучей в реальном времени на многоядерном процессоре. Высокопроизводительные параллельные вычисления на кластерных системах (HPC-2008). Материалы Восьмой Международной конференции - семинара. Казань, ноябрь 17-19, 2008. Труды конференции -- Казань: Изд. КГТУ, 2008. - С. 241.
\item Морозов А. С. Высокопроизводительная реализация трассировки лучей с использованием Microsoft MPI. Технологии Microsoft в теории и практике программирования. Материалы конференции / Под ред. Проф. В.П. Гергеля. -- Нижний Новгород: Изд-во Нижегородского госуниверситета, 2009. - 527 с.
\item Морозов А. С. Сравнительный анализ алгоритма трассировки лучей на системах с общей и разделяемой памятью. Параллельные вычислительные технологии (ПаВТ’2009): Труды международной научной конференции (Нижний Новгород, 30 марта - 3 апреля 2009 г.). -- Челябинск: Изд. ЮурГУ, 2009. - 839 с
\end{enumerate}

\vspace*{5cm}
{Научный руководитель: \noindent
\hspace*{1.5cm}\hspace{-1.5cm}{\ \ \hrulefill\ \ \ Турлапов В. Е. }
}

\end{document}