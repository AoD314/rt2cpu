\subsection{Модели освещения}

В соответствии с принятым в компьютерной графике подходом, расчет освещенности распадается на две основные задачи. Первая – определить способ расчета освещенности в произвольной точке трехмерного пространства, решается при помощи построения обобщенной математической модели освещенности (Illuminating model). Вторая задача – применение Illuminating model для компьютерных расчетов освещенности трехмерных объектов с конкретной геометрией и свойствами поверхности, решается при помощи так называемой модели затенения (Shading model).

Моделей освещенности к настоящему моменту разработано несколько. Самая первая,и самая простая – локальная модель освещенности. Эта модель не рассматривает процессы светового взаимодействия объектов сцены между собой, а только расчет освещенности самих объектов. Вторая, глобальная модель освещенности – Global Illuminations, рассматривает трехмерную сцену как единую систему и пытается описывать освещение с учетом взаимного влияния объектов. В рамках этой модели рассматриваются такие вопросы, как многократное отражение и преломление света (ray tracing), рассеянное освещение (radiosity), каустик(caustic) и фотонные карты (photon mapping) и другие. 

\subsubsection{Глобальные модели освещения}

Глобальное освещение (global illumination) — это название ряда алгоритмов, используемых в 3D-графике, которые предназначены для добавления более реалистичного освещения в трёхмерные сцены. Такие алгоритмы учитывают не только свет, который поступает непосредственно от источника света (прямое освещение, англ. direct illumination), но и такие случаи, в которых лучи света от одного и того же источника, отражаются на других поверхностях сцены (непрямая освещенность, англ. indirect illumination).

Теоретически отражение, преломление, тень — примеры глобального освещения, потому что, для их имитации необходимо учитывать влияние одного объекта на другие (в отличие от случая когда на объект падает прямой свет). На практике, однако, только моделирование диффузного отражения или каустики называется глобальным освещением.

Изображения полученные в результате применения алгоритмов глобального освещения часто кажутся более фотореалистичными, чем те, в процессе рендеринга которых применялись алгоритмы только прямого освещения, но для просчета глобального освещения требуется гораздо больше времени.

\subsubsection{Локальные модели освещения}

Существующие локальные модели освещения можно разделить на две категории. К первой категории относятся эмпирические модели. Они обычно эффективны в плане быстродействия и некоторые из них дают довольно реалистичную картинку. Они обычно не оперируют такими физическими величинами, как световая энергия, или световой поток. Однако эти модели находят довольно широкое применение в областях, где не требуется точная физическая информация об освещении (например, спецэффекты в фильмах, программы для художников и дизайнеров, для рекламных целей)

Ко второй категории относятся модели, базирующиеся на физических представлениях о теории света. Изображения, полученные с использованием этих моделей, очень хорошо соотносятся с экспериментальными данными. Поэтому эти модели находят применение там, где важна точная имитация поведения света (оформление интерьеров, архитектура)

\subsubsection{Модель Фонга}

Это эмпирическая модель. В самом общем случае, в свете требования фотореалистичности, эта модель учитывает и неявное ambient-освещение. Ambient-освещение, или его еще называют фоновым (background), – это окружающее объект освещение от удаленных источников, чье положение и характеристики не известны. Необходимость учета ambient-освещения, пусть и очень грубо, обусловлена тем, что его вклад может быть достаточно велик – до 50\% от общей освещенности. В Local Illumination считают, что фоновое освещение задает цвет (и его интенсивность) объекта в отсутствии явных источников света или в тени. Не несет никакой информации об объекте, кроме значения простого цвета, равномерно заливающего контур объекта.

Интенсивность такого освещения постоянна и равномерно распределена во всем пространстве, расчет его отражения поверхностью выполняется по формуле:

$$
 \vec{I}_{amb} = K_{a} \cdot \vec{I}_{a}
$$

где $\vec{I}_{amb}$  - интенсивность отраженного ambient освещения, $K_{a}$ - коэффициент, характеризующий отражающие свойства поверхности для  ambient-освещения, $\vec{I}_{a}$ - исходная интенсивность ambient-света, падающего на поверхность.

Часть света от прямых источников зеркально отражается поверхностью, а остальной свет диффузно рассеивается во всех направлениях. Кроме чисто зеркального отражения, которое имеют идеально отполированные поверхности, различают так называемое glossiness или распределенное зеркальное отражение – отражение в некотором створе углов, а не на один единственный угол. Такое рассеяние света обусловлено микрорельефом ("шероховатостью") поверхности, то есть поверхность реальных объектов не является идеально гладкой, а состоит из большого количества микровыступов и впадин, которые зеркально отражают падающий свет под разными углами. Результатом glossy-отражения является specular highlight – яркий световой блик, имеющий размер в зависимости от степени шероховатости поверхности. 

Интенсивность рассеянного света зависит от угла падающего на поверхность света по закону Ламберта (Lambert):

$$
 \vec{I}_{diff} = K_{diff} \cdot \vec{I}_{d} \cdot \cos(\alpha)
$$
 
где $\vec{I}_{d}$ - интенсивность падающего на поверхность света, $K_{diff}$ - коэффициент, характеризующий рассеивающие свойства поверхности, $\cos(\alpha)$ - угол между направлением на источник света и нормалью поверхности

Другими словами, поверхность будет освещена больше, если свет падает на нее перпендикулярно ($\alpha = 0$), и меньше, если свет падает под любым другим углом, поскольку в этом случае увеличивается освещаемая площадь. Диффузно рассеянный свет является главным источником визуальной информации о геометрии трехмерных объектов.

Как было уже сказано ранее, свет отражается зеркально в некотором створе углов, и для большинства реальных материалов мы всегда видим зеркальную подсветку в форме светового пятна, а не в форме яркой точки. Поэтому, для расчета интенсивности зеркально отраженного света используется формула, предложенная Фонгом:

$$
 \vec{I}_{spec} = K_{spec} \cdot \vec{I}_{s} \cdot \cos^n(\beta)
$$

где $\vec{I}_{spec}$ - интенсивность зеркально отраженного света,  $\vec{I}_{s}$ - интенсивность источника света,  $\vec{K}_{s}$ - коэффициент, характеризующий свойства зеркального отражения поверхности
$\beta$ - угол между направлением идеального отражения и направлением на наблюдателя, степень $n$ определяет размер пятна светового блика, чем больше $n$, тем меньше световой блик, и тем ближе отражающие свойства поверхности к свойствам идеального зеркала.

Формула Фонга – пример компьютерной фикции, поскольку она не имеет физического смысла. Ее используют просто потому, что она дает хорошие практические результаты. 

Таким образом, локальная модель освещенности предполагает расчет отраженной фоновой освещенности, диффузного и зеркального отражения от прямых источников: 

$$
 \vec{I}_{local} =  K_{amb} \cdot \vec{I}_{amb} +  K_{diff} \cdot \vec{I}_{diff} \cdot \left( \vec{L},\vec{N} \right) + K_{spec} \cdot \vec{I}_{spec} \cdot \left( \vec{R},\vec{V} \right)^n
$$
