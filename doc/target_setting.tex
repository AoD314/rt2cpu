\newpage 
\section{Постановка задачи}
Главной целью данной работы является разработка и исследование алгоритма трассировки лучей на архитектуре x64 с применением ускоряющей структуры. Для решения главной задачи, требуется решить ряд следующих подзадач:
\begin{itemize}
	\item Реализовать высокопроизводительный алгоритм трассировки лучей на центральном процессоре
	\item Реализация и исследование оптимизированной версии с использованием векторных расширений архитектуры x64
	\item Реализация и исследование специализированного класса векторов для алгоритма трассировки лучей основанного на технологии шаблонных выражений, с применением векторных оптимизаций  - \index{SIMD}SIMD\footnote{ Single Instruction, Multiple Data — Одна Инструкция, Много Данных } инструкции 
	\item Реализация параллельной версии алгоритма трассировки лучей с использованием OpenMP, TBB
	\item Сравнение параллельной версии алгоритма трассировки лучей с использованием библиотеки TBB и расширения языка OpenMP
	\item Реализация ускоряющей структуры
	\item Сравнение реализации алгоритма с использованием ускоряющей структуры и без нее
\end{itemize}
   В качестве основного языка программирования выбирается язык С++, а для отображения результатов — кроссплатформенная библиотека SDL. Таким образом данное программное обеспечение сможет работать как на машине с операционной системой Windows, так и с операционной системой GNU/Linux.
   