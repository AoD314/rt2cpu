\documentclass[utf8, 12pt]{beamer}
\usepackage[utf8]{inputenc}
\usepackage[russian]{babel}

\usepackage{eulervm}

%выбор темы
\usetheme{Madrid}
\useoutertheme{shadow}
\beamertemplatenavigationsymbolsempty
\setbeamertemplate{footline}[frame number]

\usepackage{fontspec}
\defaultfontfeatures{Scale=MatchLowercase}
\setmainfont[Mapping=tex-text]{Liberation Serif}
\setsansfont[Mapping=tex-text]{Liberation Sans}
\setmonofont{Droid Sans Mono}

% переопределяем окружение verbatim
\makeatletter 
\renewcommand*\verbatim@font{% 
\normalfont\ttfamily\small 
\hyphenchar\font\m@ne 
\let\do\do@noligs \verbatim@nolig@list 
} 
\makeatother 

\title{Трассировка лучей в реальном времени \\ на~x64~архитектуре}
\author{}\date{}

\begin{document}
\begin{frame}
\begin{center}
{\tiny
Министерство образования и науки Российской Федерации\\ 
Государственное образовательное учреждение \\ 
высшего профессионального образования \\ 
<<Нижегородский государственный университет им. Н.И. Лобачевского>>\\
\bf{Факультет вычислительной математики и кибернетики \\
Кафедра математического обеспечения ЭВМ} \\
}
\end{center}
\titlepage
\vspace*{-1cm}
\begin{flushright}
\begin{tabular}{rl}
Исполнитель: & Морозов А.С. \\
Научный руководитель: & Турлапов В.Е.
\end{tabular}
\end{flushright}
\vspace*{1.75cm}
\begin{center}
\small Нижний Новгород, 2011г.
\end{center}
\end{frame}

\begin{frame}
\frametitle{Постановка задачи}
Высокопроизводительная реализация алгоритма трассировки лучей:
\begin{itemize}
\item векторизация (SIMD)
\item параллелизация (TBB, OpenMP)
\item алгоритмическая оптимизация
\end{itemize}
\end{frame}

\begin{frame}
\frametitle{Трассировка лучей}
\begin{itemize}
\item Алгоритм обратной трассировки лучей
\begin{itemize}
\item Модель Фонга
$$
 \vec{I}_{local} =  K_{amb} \cdot \vec{I}_{amb} +  K_{diff} \cdot \vec{I}_{diff} \cdot \left( \vec{L},\vec{N} \right) + K_{spec} \cdot \vec{I}_{spec} \cdot \left( \vec{R},\vec{V} \right)^n
$$
$$
 \vec{I}_{total} = \vec{I}_{local} + K_{reflection} \cdot \vec{I}_{reflection} + K_{refraction} \cdot \vec{I}_{refraction}
$$
\item Теневые лучи
\item Расчет отражений
$$
\vec{R} = \vec{I} - 2 \cdot \vec{N} (\vec{N} , \vec{I})
$$
\end{itemize}
\end{itemize}
\end{frame}


\begin{frame}
\frametitle{Оптимизация}
\end{frame}

\begin{frame}
\frametitle{Результаты}
\end{frame}

\frame[containsverbatim]{
\frametitle{Результаты}
}

\begin{frame}
\frametitle{Результаты}

\end{frame}

\begin{frame}
\frametitle{Выводы}
\end{frame}

\begin{frame}
\frametitle{Вопросы ?}
\begin{center}
{\huge Спасибо за внимание !}
\end{center}
\end{frame}


\end{document}
