\newpage
\section{Оптимизация}
Оптимизация -- как способ программирования по уровням архитектуры всерху вниз.

\subsection{Шаблоны C++}

\subsubsection{Понятие шаблона}

Шаблоны(Templates) были введены в язык C++ как средство, позволяющие параметризовать типы данных. Это связано с тем, что для классов или функций приходилось реализовывать одни и те же алгоритмы, но для разных типов данных. Получали дублирование кода, и тем самым росло число ошибок. 
Пример. Реализовать функцию, которая возвращает максимальное значение из 2 чисел.
\cppsource{src/deftemplate.tex}{Несколько реализация функции {\bf max}}
\noindent
и так далее. Приходится писать один и тот же код несколько раз. Во второй функции можно было допустить ошибку (например указать неправильный знак сравнения), которую потом очень трудно найти. Или наоборот, после обнаружения ошибки, придется править код во всех реализациях функции max (возможна ситуация, когда в нескольких местах ошибка была исправлена, а в остальных пропущена или забыта). С этими проблемами помогли справится шаблоны, которые параметризовали типы данных следующим образом:
\cppsource{src/definetemplate.tex}{Шаблонное определение функции {\bf max}}
%\noindent
Т. о. работу которую выполнял программист теперь выполняет компилятор. При вызове функции, в качестве параметров которых нужно сравнить два int, компилятор сам из шаблона выведет функцию max(int,int).
\subsubsection{Вычисление на шаблонах}
Сегодня шаблоны используют различным образом, не так как ожидали изобретатели
шаблонов С++. Сегодня программирование на шаблонах включают различные техники, такие как: обобщенное программирование, вычисление во время компиляции, шаблонные выражения(expression templates), мета-программирование, и др.

\noindent Рассмотрим пример вычисления факториала. \\
\noindent Факториал числа N это: $N! = N \cdot (N-1) \cdot \dots \cdot 1$

Рекурсивная реализация факториала, без использование шаблонов, приведена в следующем листинге:
\cppsource{src/rekfactorial.tex}{Рекурсивная реализация факториала}
Эту функцию следует использовать следующим образом:
\begin{verbatim}
cout << factorial(7) << endl;
\end{verbatim}
Вызывать рекурсивно функцию - это очень большие накладные расходы. Несмотря на то, что мы указало компилятору встроить функцию (inline), компилятор проигнорирует это, так как он не может сделать постановку в рекурсию. Можно добиться большего успеха, если реализовывать это, как класс с шаблоном.
\cppsource{src/templatefactorial.tex}{Реализация факториала на шаблонах}
Можно заметить, что у данного шаблона нет ни данных, ни функциональных участков, это только определение перечислимого типа. Для того чтобы можно было определить шаблон для n, нужно для начала определить шаблон для n-1, т. е. для n-2, n-3 и т. д. В итоге получаем рекурсию. Следует заметить, что в качестве параметра шаблона используется обычный тип int. В нашем случае есть параметр шаблона типа int, это означает, что в этот шаблон будет подставлено постоянное число типа int. Что бы воспользоваться данным классом необходимо написать следующие:
\begin{verbatim}
cout << factorial<7>::ret << endl;
\end{verbatim}
Компилятор рекурсивно определяет значение факториала<7>, затем <6> и так далее. Так как это рекурсия, то что бы не зациклится необходимо вовремя остановится. Любая рекурсия нуждается в остановки, и это не исключение. Это можно сделать с помощью специализации шаблона(т.е. определение для частного случая).


%
%%\par \textcolor{black}{TEXT text text text text text text text }
%%\par \textcolor{red}{TEXT text text text text text text text }
%%\par \textcolor{blue}{TEXT text text text text text text text }
%%\par \textcolor{magenta}{TEXT text text text text text text text }
%%\par \textcolor{green}{TEXT text text text text text text text }
%%\par \textcolor{cyan}{TEXT text text text text text text text }
%%\par \textcolor{yellow}{TEXT text text text text text text text }
%%\par
%%\par \textcolor{GreenYellow}{TEXT text text text text text text text }
%%\par \textcolor{Yellow}{TEXT text text text text text text text }
%%\par \textcolor{Goldenrod}{TEXT text text text text text text text }
%%\par \textcolor{Dandelion}{TEXT text text text text text text text }
%%\par \textcolor{Apricot}{TEXT text text text text text text text }
%%\par \textcolor{Peach}{TEXT text text text text text text text }
%%\par \textcolor{Melon}{TEXT text text text text text text text }
%%\par \textcolor{YellowOrange}{TEXT text text text text text text text }
%%\par \textcolor{Orange}{TEXT text text text text text text text }
%%\par \textcolor{BurntOrange}{TEXT text text text text text text text }
%\par \textcolor{Bittersweet}{TEXT text text text text text text text }
%\par \textcolor{RedOrange}{TEXT text text text text text text text }
%\par \textcolor{Mahogany}{TEXT text text text text text text text }
%\par \textcolor{Maroon}{TEXT text text text text text text text }
%\par \textcolor{BrickRed}{TEXT text text text text text text text }
%\par \textcolor{Red}{TEXT text text text text text text text }
%\par \textcolor{OrangeRed}{TEXT text text text text text text text }
%\par \textcolor{RubineRed}{TEXT text text text text text text text }
%\par \textcolor{WildStrawberry}{TEXT text text text text text text text }
%\par \textcolor{Salmon}{TEXT text text text text text text text }
%\par \textcolor{CarnationPink}{TEXT text text text text text text text }
%\par \textcolor{Magenta}{TEXT text text text text text text text }
%\par \textcolor{VioletRed}{TEXT text text text text text text text }
%\par \textcolor{Rhodamine}{TEXT text text text text text text text }
%\par \textcolor{Mulberry}{TEXT text text text text text text text }
%\par \textcolor{RedViolet}{TEXT text text text text text text text }
%\par \textcolor{Fuchsia}{TEXT text text text text text text text }
%\par \textcolor{Lavender}{TEXT text text text text text text text }
%\par \textcolor{Thistle}{TEXT text text text text text text text }
%\par \textcolor{Orchid}{TEXT text text text text text text text }
%\par \textcolor{DarkOrchid}{TEXT text text text text text text text }
%\par \textcolor{Purple}{TEXT text text text text text text text }
%\par \textcolor{Plum}{TEXT text text text text text text text }
%\par \textcolor{Violet}{TEXT text text text text text text text }
%\par \textcolor{RoyalPurple}{TEXT text text text text text text text }
%\par \textcolor{BlueViolet}{TEXT text text text text text text text }
%\par \textcolor{Periwinkle}{TEXT text text text text text text text }
%\par \textcolor{CadetBlue}{TEXT text text text text text text text }
%\par \textcolor{CornflowerBlue}{TEXT text text text text text text text }
%\par \textcolor{MidnightBlue}{TEXT text text text text text text text }
%\par \textcolor{NavyBlue}{TEXT text text text text text text text }
%\par \textcolor{RoyalBlue}{TEXT text text text text text text text }
%\par \textcolor{Blue}{TEXT text text text text text text text }
%\par \textcolor{Cerulean}{TEXT text text text text text text text }
%\par \textcolor{Cyan}{TEXT text text text text text text text }
%\par \textcolor{ProcessBlue}{TEXT text text text text text text text }
%\par \textcolor{SkyBlue}{TEXT text text text text text text text }
%\par \textcolor{Turquoise}{TEXT text text text text text text text }
%\par \textcolor{TealBlue}{TEXT text text text text text text text }
%\par \textcolor{Aquamarine}{TEXT text text text text text text text }
%\par \textcolor{BlueGreen}{TEXT text text text text text text text }
%\par \textcolor{Emerald}{TEXT text text text text text text text }
%\par \textcolor{JungleGreen}{TEXT text text text text text text text }
%\par \textcolor{SeaGreen}{TEXT text text text text text text text }
%\par \textcolor{Green}{TEXT text text text text text text text }
%\par \textcolor{ForestGreen}{TEXT text text text text text text text }
%\par \textcolor{PineGreen}{TEXT text text text text text text text }
%\par \textcolor{LimeGreen}{TEXT text text text text text text text }
%\par \textcolor{YellowGreen}{TEXT text text text text text text text }
%\par \textcolor{SpringGreen}{TEXT text text text text text text text }
%\par \textcolor{OliveGreen}{TEXT text text text text text text text }
%\par \textcolor{RawSienna}{TEXT text text text text text text text }
%\par \textcolor{Sepia}{TEXT text text text text text text text }
%\par \textcolor{Brown}{TEXT text text text text text text text }
%\par \textcolor{Tan}{TEXT text text text text text text text }
%\par \textcolor{Gray}{TEXT text text text text text text text }
%\par \textcolor{Black}{TEXT text text text text text text text }
%\par \textcolor{White}{TEXT text text text text text text text }
%\par

