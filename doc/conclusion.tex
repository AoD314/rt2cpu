\newpage
\section*{Заключение}
\addcontentsline{toc}{section}{Заключение} 
Задача трассировки лучей является по настоящему трудным испытанием для центрального процессора. Несмотря на то, что процессор обладает хорошей производительностью на ядро, общей производительности ему не хватает. Несмотря на столь малые мощности, удалось реализовать достаточно быстрый алгоритм на центральном процессоре. Для большей производительности была разработана эффективная параллельная версия программы с использованием библиотеки TBB и
стандарта параллельного OpenMP. Благодаря использованию языка с++ и технику шаблонных выражений, удалось еще повысить производительность программы. Программа продемонстрировала хорошую производительность: используя всего лишь один процессор можно получать изображения в реальном времени.

