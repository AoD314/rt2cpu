
\subsection{Примитивы}
\subsubsection{\index{Плоскость}Плоскость}
Для определения пресечения луча с плоскостью, необходимо найти точку в пространстве, которая будет удовлетворять двум уравнениям: уравнению луча и уравнению плоскости.
\par
Уравнение луча:
\begin{equation}
\left\{
\begin{array}{ccccc}
x & = & x_0 & + & t \cdot x_d \\
y & = & y_0 & + & t \cdot y_d \\
z & = & z_0 & + & t \cdot z_d \\
\end{array}
\right.
\label{equation_ray}
\end{equation}
или
$$
\vec{R}(t) = \vec{O} + t \cdot \vec{D}
$$
где $O = \VECXYZ{0} $ - начало луча, а $D = \VECXYZ{d} $ - направление луча. 
\par
Уравнение плоскости задается следующим образом:
\begin{equation}
Ax + By + Cz + D = 0
\label{equation_plane}
\end{equation}
\par
Для того, что бы найти точку пересечения луча с плоскостью, необходимо подставить уравнение (\ref{equation_ray}) в (\ref{equation_plane}):
$$
A(x_0 + t \cdot x_d) + B(y_0 + t \cdot y_d) + C(z_0 + t \cdot z_d) + D = 0
$$
Раскроем скобки и приведем подобные
$$
t(Ax_d + By_d + Cz_d ) + Ax_0 + By_0 + Cz_0 + D = 0
$$
найдем неизвестную величину t
$$
 t = - \frac{Ax_0 + By_0 + Cz_0 + D}{Ax_d + By_d + Cz_d}
$$
из уравнения видно, что луч либо пересекает плоскость в какой то точке, либо нет. Это связано с тем, что если $Ax_d + By_d + Cz_d = 0$, то плоскость и луч параллельны друг другу. Т.к. $P = \VEC A B C $ - это нормаль к поверхности, то из геометрии известно, что если $( D , P ) = 0$, то вектора параллельны.
\par
   Для того, что бы найти величину $t$, необходимо рассчитать всего несколько скалярных произведений:
$$
t = - \frac{(O, P) + D}{(D, P)}
$$
при условии, что $(D, P) \neq 0$

\subsubsection{\index{Сфера}Сфера}
Для сферы необходимо проделать те же выкладки.
Уравнение луча:
\begin{equation}
\left\{
\begin{array}{ccccc}
x & = & x_0 & + & t \cdot x_d \\
y & = & y_0 & + & t \cdot y_d \\
z & = & z_0 & + & t \cdot z_d \\
\end{array}
\right.
\label{equation_ray}
\end{equation}
или
$$
\vec{R}(t) = \vec{O} + t \cdot \vec{D}
$$
Уравнение сферы записывается следующем образом:
\begin{equation}
(x-x_c)^2 + (y-y_c)^2 + (z-z_c)^2 = r^2
\label{equation_sphere}
\end{equation}
где $S = \VECXYZ{c}$ - центр сферы, а $r$ - радиус. Подставим уравнение (\ref{equation_ray}) в (\ref{equation_sphere}):
$$
\left( (x_0 + t \cdot x_d) - x_c \right)^2 + 
\left( (y_0 + t \cdot y_d) - y_c \right)^2 + 
\left( (z_0 + t \cdot z_d) - z_c \right)^2 = r^2
$$
приведем это уравнение в виду
\begin{equation}
A \cdot t^2 + B \cdot t + C = 0
\label{equation_square}
\end{equation}
после раскрытия скобок и приведения подобных, получаем:
\par
$$
A = x^2_d + y^2_d + z^2_d
$$ 
\par
$$
B = 2x_d(x_0 - x_c) + 2y_d(y_0 - y_c)  + 2z_d(z_0 - z_c) 
$$ 
\par
$$
C = (x_0 - x_c)^2 + (y_0 - y_c)^2 + (z_0 - z_c)^2 
$$
Если уравнение (\ref{equation_square}) не имеет вещественных решений, то луч не пересекает сферу. Если имеется два решения, то наименьший положительный корень этого уравнения определит на луче ближайшую точку пересечения луча со сферой.
\par
Рассмотрим подробнее как вычисляются коэффициенты $A$, $B$, $C$:
\par
$$
A = x^2_d + y^2_d + z^2_d = (D, D)
$$ 
\par
$$
B = 2x_d(x_0 - x_c) + 2y_d(y_0 - y_c)  + 2z_d(z_0 - z_c)  = 2 \cdot (D, D - S)
$$ 
\par
$$
C = (x_0 - x_c)^2 + (y_0 - y_c)^2 + (z_0 - z_c)^2 = (O - S, O - S )
$$
Далее решаем обыкновенное квадратное уравнение и находим корни и получаем значение $t$
$$
t_{1,2} = \frac{-B \pm \sqrt{B^2 - 4 \cdot A \cdot C}}{2 \cdot A}
$$

$$
t_{1} = \frac{ - 2 \cdot (\vec{D}, \vec{D} - \vec{S}) + \sqrt{(2 \cdot (\vec{D}, \vec{D} - \vec{S}))^2 - 4 \cdot (\vec{D}, \vec{D}) \cdot (\vec{O} - \vec{S}, \vec{O} - \vec{S} )}}{2 \cdot (\vec{D}, \vec{D})}
$$

$$
t_{2} = \frac{ - 2 \cdot (\vec{D}, \vec{D} - \vec{S}) - \sqrt{(2 \cdot (\vec{D}, \vec{D} - \vec{S}))^2 - 4 \cdot (\vec{D}, \vec{D}) \cdot (\vec{O} - \vec{S}, \vec{O} - \vec{S} )}}{2 \cdot (\vec{D}, \vec{D})}
$$

\subsubsection{\index{Треугольник}Треугольник}
Алгоритм пересечения луча и треугольника основан на барицентрических координатах.
\par
Барицентрические координаты -- координаты точки $n$-мерного аффинного пространства $A^n$, отнесенные к некоторой фиксированной системе из $(n + 1)$-ой точки $p_0, p_1, \dots, p_n$ , не лежащих в $(n -1)$-мерном подпространстве. Пусть $z$ есть произвольная точка в $A^n$. Каждая точка $x \in A^n$ может быть единственным образом представлена в виде суммы
$$
	x = z + \alpha_1 \cdot z\vec{p_1} + \alpha_2 \cdot z\vec{p_2} + \cdots + \alpha_n \cdot z\vec{p_n} 
$$
где $\alpha_1, \alpha_2, \dots,  \alpha_n $ вещественные числа, удовлетворяющие условию
$$
\alpha_1 + \alpha_2 + \cdots + \alpha_n  = 1
$$
Числа $\alpha_1, \alpha_2, \dots,  \alpha_n $ называются барицентрическими координатами точки $x$. Легко видеть, что барицентрические координаты не зависят от выбора $z$.

Точка $T(u,v)$, принадлежащая треугольнику, может быть записана в виде:
\begin{equation}
\label{triangle_bar}
T(u,v) = (1-u-v)V_0 + uV_1 + vV_2
\end{equation}

\noindent где $(u,v)$ -- это бариецентрические координаты такие, что $u \geq 0$, $v \geq 0$ и $u + v \leq 1$

Вычисление пересечения между лучем(\ref{equation_ray}) и треугольником(\ref{triangle_bar}), это решение следующего уравнения:

$$
O + tD = (1-u-v)V_0 + uV_1 + vV_2
$$

\noindent после нескольких очевидных преобразований:

$$
O + tD = V_0 - uV_0 - vV_0 + uV_1 + vV_2
$$
$$
O - V_0 = - tD + uV_1 - uV_0 + vV_2 - vV_0
$$

$$
- tD + u(V_1 - V_0) + v(V_2 - V_0) = O - V_0
$$

\noindent получаем:

\begin{equation}
\label{ray_cross_tr}
\left[ -D, V_1 - V_0 , V_2 - V_0\right] 
\left[  
\begin{array}{c} t \\ u \\ v \\ \end{array}
\right] = O - V_0
\end{equation}

Что бы решить задачу, необходимо найти вектор $\VEC{t}{u}{v}$. Обозначив $E_1 = V_1 - V_0$, $E_2 = V_2 - V_0$ и $T = O - V_0$ решим уравнение (\ref{ray_cross_tr}), используя метод Крамера:

\begin{equation}
\label{ray_cross_tr_solv}
\left[  
\begin{array}{c} t \\ u \\ v \\ \end{array}
\right] = 
\frac{1}{| -D , E_1, E_2|} 
\left[  
	\begin{array}{lcccl}
	|& T , & E_1, &E_2 & | \\
	|&-D , & T  , &E_2 & | \\
	|&-D , & E_1, &T   & | \\
	\end{array}
\right] 
\end{equation}

\noindent Из курса линейной алгебры известно, что: $|A, B, C| = - (A \times C) \cdot B = - (C \times B) \cdot A$. Принимая во внимания этот факт, перепишем уравление (\ref{ray_cross_tr_solv}).

\begin{equation}
\label{ray_cross_tr_solv2}
\left[  
\begin{array}{c} t \\ u \\ v \\ \end{array}
\right] = 
\frac{1}{(D \times E_2)\cdot E_1} 
\left[  
	\begin{array}{c}
	(T \times E_1) \cdot E_2 \\
	(D \times E_2) \cdot T \\
	(T \times E_1) \cdot D \\
	\end{array}
\right] = 
\frac{1}{P \cdot E_1} 
\left[  
	\begin{array}{c}
	Q \cdot E_2 \\
	P \cdot T \\	
	Q \cdot D \\	
	\end{array}
\right]
\end{equation}
где $P = (D \times E_2)$ и $Q = T \times E_1$



