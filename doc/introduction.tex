\newpage
%\section{Введение}
\section*{Введение}
\addcontentsline{toc}{section}{Введение} 
В кино индустрии к современной компьютерной графике предъявляются серьезные требования физически корректного моделирования о освещения сцен, состоящая из множества примитивов с различными характеристиками взаимодействия со светом. Даже малейшие неточности, могут отбросить художественный или анимационный фильм в рубрику любительского кино, и при этом не принести ожидаемой прибыли. Особенные требования предъявляются именно к художественному фильму, т. к. используемые спец эффекты должны выглядеть настолько реалистично, что бы зритель не смог различить, где настоящий актер, а где рисованный двойник. Используя только физически правильные модели и алгоритмы можно обеспечить растущую потребность в более реалистичной трехмерной графике. \par
   С каждым новым фильмом, каждый из нас видит прогресс в компьютерной графике. Картинка становится все красочнее и правдоподобнее, но это все не дается просто так. Естественно, платить за это приходится высокой вычислительной трудоемкостью расчетов. Несомненно, что с каждым годом производительность вычислительной техники растет, но она сразу же «расходуются» на новые спецэффекты. Существует наблюдение, которое гласит, что время расчета одного кадра не изменяется. Среднее время расчета полного фильма 15 лет назад занимал около 10-12 месяцев, так и сегодня тратят столько же времени, хотя при этом, надо заметить, что производительность современных компьютеров в десятки, а то и в сотни раз превышает производительность компьютеров того времени. Со временем улучшается и требования к самому изображению. Если несколько лет
назад картинка с разрешением 1024х768 считалась излишеством в компьютерной графики, то уже сейчас это слишком мало и все считают де факто FullHD\footnote{ \index{FullHD}FullHD -- это разрешение экрана 1920х1080 пикселей} , хотя уже задумываются о еще лучшем качестве. В последний год компьютерная индустрия, дабы не потерять зрителя, начала использовать новые технологии — 3D, которая требует еще большей вычислительной мощности. \par
   Именно за последние несколько лет компьютеры стали по настоящему параллельными. Появились многоядерные процессоры. И именно по этому, что 15 лет назад было трудоемкой задачей рендеринга, то сейчас это можно получить почти в реальном времени при том же качестве результата.
